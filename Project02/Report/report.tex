\documentclass[10pt,conference,compsocconf]{IEEEtran}

\usepackage{hyperref}
\usepackage{graphicx}
\usepackage[]{algorithm2e}
\usepackage{placeins}

\begin{document}
\title{Machine Learning - Project 1}
\author{
 Alexander Holloway - Bastian Nanchen - Arnaud Pannatier 
  \\
  \textit{Group : }Max and Lily learn ML \\
  \textit{EPFL 2017}
}
\maketitle

\begin{abstract}
The goal of this project is to characterize events measured by CERN colliders to detect Higgs bosons which play a fundamental role in particles physics. Using machine learning methods we achieved to classify the CERN measures with an 83\% accuracy.

\end{abstract}

\section{Introduction}


\section{The Structure of a Paper}

\begin{description}
\item[ML Methods] \ \\
  The 6 traditional machine learning methods.
\item[Our approach] \ \\
  An explanation of the preprocessing we did and details on some techniques we used in our code.
\item[Procedure] \ \\
  A summary of the steps from raw data to the results.
\item[Results] \ \\
  A comment on the results and possible improvement.
\end{description}


\section{ML Methods seen in Class}
\section{Our Approach}


\section{Procedure}

\begin{algorithm}[h]
\DontPrintSemicolon
\KwData{The CSV file containing the data to classify}
 \KwResult{Predictions}
 Data is clustered into 4 groups as explained in the \textit{Preprocessing and data cleaning} subsection\;
\For{Degrees up to 13}{
Feature engineering (extending the matrix with additional degrees) \;
\For{Each of the 4 groups} {
\For{Subgroup in cross validation} {
Select best lambda and apply ridge regression\;
}
Remember the best prediction\;
}
}
Recombine the data of all groups\;
Return the prediction\;
\end{algorithm}

\section{Results}



\section{Conclusion}

\begin{thebibliography}{9}
\bibitem{CERN}
Cian O'Luanaigh, \textit{Le centre de donn�es du CERN franchit les 100 p�taoctets} \\
consulted the 28.10.17 at \url{https://home.cern/fr/about/updates/2013/02/cern-data-centre-passes-100-petabytes}
\end{thebibliography}


\end{document}


